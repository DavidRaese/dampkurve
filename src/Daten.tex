\documentclass[a4paper, 12pt]{report}

\usepackage[utf8]{inputenc}
\usepackage[ngerman]{babel}
\usepackage{chemfig}
\usepackage[version=3]{mhchem}
\usepackage{wasysym}
\usepackage{textcomp}
\usepackage{siunitx}
\usepackage{graphicx}
\usepackage{float}
\usepackage{amsmath}
%\usepackage{chemmacros}

\begin{document}
	
	\section*{Ergebnisse} 
		
		\textbf{\ce{H2O}:}
		\begin{table}[H]
			\label{tab:Wasser}
				\begin{tabular}{|l|l|}
				\hline
				\textbf{T [°C]}	&	\textbf{p [mbar]} \\\hline
				99,5	&	1006 \\
				97		&	916 \\
				94		&	823 \\
				90,5	&	718 \\
				86,7 	&	611 \\
				81,8	&	514 \\
				76,1	&	408 \\
				69,8	&	308 \\\hline
				\end{tabular}
		\end{table} 
		
		\textbf{\ce{H2O + NaCl}:}
		\begin{table}[H]
			\label{tab:NaCl}
				\begin{tabular}{|l|l|}
				\hline
				\textbf{T [°C]}	&	\textbf{p [mbar]} \\\hline
				104,3	&	1006 \\
				102,1	&	907 \\
				99,1	&	818 \\
				95,5	&	714 \\
				91,3 	&	620 \\
				85,5	&	500 \\
				80,3	&	390 \\
				75,8	&	328 \\\hline
				\end{tabular}
		\end{table} 

		\textbf{\ce{H2O + CaCl2}:}
		\begin{table}[H]
			\label{tab:CaCl2}
				\begin{tabular}{|l|l|}
				\hline
				\textbf{T [°C]}	&	\textbf{p [mbar]} \\\hline
				111,2	&	1006 \\
				109,4	&	928 \\
				107,3	&	822 \\
				103,5	&	730 \\
				99,5 	&	610 \\
				95,7	&	535 \\
				89,9	&	426 \\
				84,4	&	345 \\\hline
				\end{tabular}
		\end{table} 

\end{document}