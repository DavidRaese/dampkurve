\section*{Diskussion}
	
	\textbf{Unterschiede zwischen berechneten und gemessenen Siedepunktserhöhungen:}
	\begin{table}[H]
			\caption{Theoretische und tatsächliche Siedepunktserhöhung für \ce{NaCl}}
			\label{tab: Diskussion1}
				\begin{tabular}{|l|l|l|}
				\hline
				\multicolumn{3}{|c|}{\textbf{\ce{NaCl}}} \\\hline
				\textbf{Druck [mbar]}	 &	\textbf{$\Delta$T$_\text{real}$ [K]}	&	\textbf{$\Delta$T$_\text{theoretisch}$ [K]} \\\hline
				1006	&	4,8		&	3,46	 \\
				350		&	4,84	&	2,97 \\\hline
				\end{tabular}
		\end{table} 	
	
		\begin{table}[H]
			\caption{Theoretische und tatsächliche Siedepunktserhöhung für \ce{NaCl2}}
			\label{tab: Diskussion2}
				\begin{tabular}{|l|l|l|}
				\hline
				\multicolumn{3}{|c|}{\textbf{\ce{CaCl2}}} \\\hline
				\textbf{Druck [mbar]}	 &	\textbf{$\Delta$T$_\text{real}$ [K]}	&	\textbf{$\Delta$T$_\text{theoretischl}$ [K]} \\\hline
				1006	&	11,7		&	4,20 \\
				350		&	11,94		&	4,88 \\\hline
				\end{tabular}
		\end{table} 
		
		Der deutliche Unterschied zwischen den erwarteten und den tatsächlichen Siedepunktserhöhungen könnte daran liegen, dass wir für die Berechnung eine ideale Lösung annehmen, wo $\gamma _J = 1$ gilt, die Aktivität also genau der Konzentration entspricht. \\
		Aus der Berechnung der Aktivitätskoeffizienten geht jedoch hervor, dass der Wert von $\gamma _J > 1$ ist und folglich $a_J = \gamma _J \cdot x_J$ einen größeren Wert annimmt. \\
		\newline
		\newline
		\textbf{Molare Verdampfungsenthalpie $\Delta _VH_m$ von Wasser:} \\
		\newline
		Mit 42,1 $\pm$ 0,9 kJ/mol liegt die experimentell ermittelte durschnittliche Verdampfungsenthalpie im Bereich zwischen  40,7 kJ/mol ($\Delta _VH$, 100°C) und 44,0 kJ/mol ($\Delta _VH$, 25°C).
