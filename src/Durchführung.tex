\section*{Abstract}
		Von drei verschiedenen L"osungen (Wasser, Kochsalz- und Calciumchloridl"osung) wurden die Siedetemperaturen mit abfallendem Umgebungsdruck bestimmt. Mittels Grafik konnte die Verdampfungsenthaplie $\Delta H_V$ ermittelt werden.
	
	\section*{Durchf"uhrung}
		Die Durchführung erfolgte analog zur Vorgabe im Skriptum "Labor"ubung aus Physikalischer Chemie, "Ubung A: Dampfdruck".\\
		Der Zweihalskolben wurde mit der jeweiligen L"osung zum Sieden gebracht. Anschließend verringerte man den Druck innerhalb der Apparatur mit der Vakuumpupe und die jeweilige Siedetemperatur konnte bei sieben verschiedenen Dr"ucken abgelesen werden.
		
		\subsection*{Berechnung der Einwaagen}
			F"ur jede L"osung gilt $V_{\ce{H2O}} = 250$ ml. \\
			Sowohl die \ce{NaCl}- als auch die \ce{CaCl2}-L"osung sind 4 molal. \\
			\newline
			$b = \frac{n}{m_{\text{LM}}}$ \\
			\newline
			$n = \frac{m}{M}$ \\
			\newline
			M(\ce{NaCl}) = 58,44 g/mol \\
			M(\ce{CaCl2 * 2H2O}) = 147,02 g/mol \\
			M(\ce{H2O}) = 18,015 g/mol \\
			
			\begin{table}[H]
				\caption{Einwaagen}
				\label{tab: Einwaagen}
				\begin{tabular}{|l|l|l|l|}
					\hline
					\textbf{Salz}	&	\textbf{theoretische Einwaage [g]}	&	\textbf{reale Einwaage [g]}	&	\textbf{\ce{H2O} [ml]} \\\hline
					\ce{NaCl}	&	58,44	&	58,40	&	250 \\\hline
					\ce{CaCl2 * 2H2O}	&	147,02	&	147,36	&	214 \\\hline
					
					\end{tabular}
			\end{table} 	
