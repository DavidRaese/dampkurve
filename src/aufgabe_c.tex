\section{Aufgabe c)}
\subsection{Bestimmung der Steigung:}
Um die durchschnittliche molare Verdampfungsenthalpie zu bestimmen muss zunächst die Steigung aus 
Diagramm zwei berechnet werde.
Die Geradengleichung der Regressionsgerade ist gegeben durch:
\begin{align*}
    y = m \cdot x + c
\end{align*} 
Die Parameter $m(Steigung)=-5.060$ und $c(Achsenabschnitt)=13.5$ wurden mit dem Programm SciDAVis bestimmt. Zur berechnung 
der Steigung kann folgende Formel verwende werden:
\begin{align*}
    m = \frac{n \cdot \sum\limits^{n}_{i=1}(x_i \cdot yi) - \sum\limits^{n}_{i=1}x_i \cdot \sum\limits^{n}_{i=1} \cdot y_i} 
    {n \cdot \sum\limits^{n}_{i=1}x_i^2-(\sum\limits^{n}_{i=1}x_i)^2}
\end{align*} 
Der Zusammenhang der Steigung und der durchschnittlichen molaren Verdampfungsenthalpie wird über die 
Clausius-Claperyron-Gleichung hergestellt:
\begin{align*}
    \frac{d ln (p)}{d(1/T)} = \frac{-\Delta_V H_m}{R}
\end{align*}
Da bei Diagramm zwei $ln(p)$ gegen $1/T$ aufgetragen ist, entspricht $\frac{-\Delta_V H_m}{R}=m$, da m und R bekannt
sind kann $\Delta_V H_m$ bestimmt werden.
\begin{align*}
    &m = \frac{-\Delta_V H_m}{R} \\
    \Delta_V H_m = - m R = - (-5060 &K \cdot mol) \cdot 8,13 J \cdot K^{-1}  mol^{-1} = 42071J  
\end{align*} 
Die durchschnittliche Verdampfungsenthalpie die gemessen wurde beträgt somit 42071J.