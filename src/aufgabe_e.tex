\section{Aufgabe e)}
\subsection{Berechnung des Aktivitätskoeffizienten:}
Wie in Aufgabe d) zu erkennen ist, weichen die realen von den idealen
Siedetemperaturen ab. Dies liegt daran, das bei der Clausius-Claperyron-Gleichung von idealen verhalten
ausgegangen wird, es werden also nicht die Wechselwirkungen, welche zwischen den Molekülen stattfinden berücksichtigen.
Um diese Diskrepanz zu quantifizieren wird der Aktivitätskoeffizient herangezogen, dieser beschreibt das Verhältnis
zwischen idealem und realen Verhalten.
\begin{align*}
	\Delta T_{real} & = \Delta T_{ideal} \cdot \gamma            \\
	\gamma          & = \frac{\Delta T_{real}}{\Delta T_{ideal}}
\end{align*}

\begin{table}[H]
	\centering
	\caption{Tabelle}
	\begin{tabular}{lccc}
		\toprule
		                  & $\Delta T_{real}$[K] & $\Delta T_{theo.}$[K] & $\gamma$ \\
		\midrule
		NaCl 350mbar      & 4.8                  & 3.05                  & 1.6      \\
		NaCl 1006mbar     & 4.8                  & 3.54                  & 1.4      \\
		$CaCl_2$ 350mbar  & 11.9                 & 4.3                   & 2.8      \\
		$CaCl_2$ 1006mbar & 11.7                 & 5.00                  & 2.34     \\
		\bottomrule
	\end{tabular}
\end{table}

%-----------------------------------

 \section*{Berechnung der Aktivitätskoeffizienten}
	
	$\gamma_{\pm} = \frac{\Delta T_{\text{praktisch}}}{\Delta T_{\text{theoretisch}}}$ \\
	\newline
	\begin{table}[H]
		\caption{Aktivit"atskoeffizienten f"ur \ce{NaCl}}
		\label{tab: Aktivitätskoeffizient_NaCl}
			\begin{tabular}{|l|l|l|l|}
			\hline
			\multicolumn{4}{|c|}{\textbf{\ce{NaCl}}} \\\hline
			\textbf{Druck [mbar]}	 &	\textbf{real [K]}	&	\textbf{theoretisch [K]}	&	\textbf{$\gamma$} \\\hline
			1006	&	4,8		&	3,46	&	1,39 \\
			350		&	4,84	&	2,97	&	1,63 \\\hline
			\end{tabular}
	\end{table} 	

	\begin{table}[H]
		\caption{Aktivit"atskoeffizienten f"ur \ce{CaCl2}}
		\label{tab: Aktivitätskoeffizient_CaCö2}
			\begin{tabular}{|l|l|l|l|}
			\hline
			\multicolumn{4}{|c|}{\textbf{\ce{CaCl2}}} \\\hline
			\textbf{Druck [mbar]}	 &	\textbf{real [K]}	&	\textbf{theoretisch [K]}	&	\textbf{$\gamma$} \\\hline
			1006	&	11,7		&	4,20	&	2,79 \\
			350		&	11,94		&	4,88	&	2,45 \\\hline
			\end{tabular}
	\end{table} 	
