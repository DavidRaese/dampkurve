\section*{Berechnung der theoretischen Siedepunktserhöhung}
	$\Delta _V H = 41,1 \cdot 10^3$ J/mol \\
	$R = 8,3145$ J/(mol$\cdot$K) \\
	\newline
	n(\ce{H2O}) = 13,877 mol \\
	T$_{350\text{mbar}}$ = 72,54 °C = 345,69 K \\
	T$_{1006\text{mbar}}$ = 99,5 °C = 372,65 K \\
	\newline
	$\Delta T = \frac{R \cdot T^2}{\Delta _{V}H} \cdot x_{B}$ \\
	\newline
	\newline
	\underline{\textbf{NaCl:}} \\
	\newline
	$x_B = \frac{2 \textrm{ mol}}{(2 + 13,877)\textrm{mol}} = 0,126$\\
	\newline
	\underline{\underline{$\Delta \text{T}_{350 \text{ mbar}} = 3,05$ K}} \\ 
	\newline
	\underline{\underline{$\Delta \text{T}_{1006 \text{ mbar}} = 3,54$ K}} \\
	\newline
	\newline
	\underline{\textbf{\ce{CaCl2}:}} \\
	\newline
	$x_B = \frac{3 \textrm{mol}}{(3 + 13,877)\textrm{ mol}} = 0,178$\\
	\newline
	\underline{\underline{$\Delta \text{T}_{350 \text{ mbar}} = 4,30$ K}} \\ 
	\newline
	\underline{\underline{$\Delta \text{T}_{1006 \text{ mbar}} = 5,00$ K}} \\	

	\begin{table}[H]
		\caption{Theoretische Siedepunktserhöhung für \ce{NaCl}}
		\label{tab:Siedepunkt_NaCl}
			\begin{tabular}{|l|l|l|l|}
			\hline
			\textbf{Druck [mbar]}	&	\textbf{x$_\text{B}$}	&	\textbf{T* [°C]} & \textbf{$\Delta$ T [°C]} \\\hline
			1006	&	0,126	&	104,3	&	3,54 \\
			350		&	0,126	&	77,38	&	3,05 \\\hline
			\end{tabular}
	\end{table} 
	
	\begin{table}[H]
		\caption{Theoretische Siedepunktserhöhung für \ce{CaCl2}}
		\label{tab:Siedepunkt_CaCl2}
			\begin{tabular}{|l|l|l|l|}
			\hline
			\textbf{Druck [mbar]}	&	\textbf{x$_\text{B}$}	&	\textbf{T* [°C]} & \textbf{$\Delta$ T [°C]} \\\hline
			1006	&	0,178	&	111,2	&	4,88 \\
			350		&	0,178	&	84,48	&	4,20 \\\hline
			\end{tabular}
	\end{table} 		
